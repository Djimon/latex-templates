\documentclass[a4paper,12pt]{book}

\usepackage{amsmath,amsfonts,amssymb,bbm}
\usepackage{graphicx}
\usepackage{phdthesis}
\usepackage{natbib}
\usepackage{ifdraft}
\usepackage{xspace,xcolor}
\usepackage{ellipsis,fixltx2e,microtype}
\usepackage[labelfont=bf,justification=RaggedRight,textfont=it]{caption}
\usepackage{supertabular}
\usepackage{footnote}        % Fusznoten in Tabellen-Patch
\usepackage[english]{babel}
\usepackage{multirow}
\usepackage{varioref}
\usepackage{rotating}
\usepackage{booktabs}

\usepackage{fourier}         % Schrift
\usepackage{scalefnt}
\usepackage[scaled=0.875]{helvet} % ss
\renewcommand{\ttdefault}{lmtt} %tt

\usepackage{pifont}   % \ding{} commands

\usepackage{makeidx}
\usepackage[columns=2]{idxlayout}

\usepackage[linesnumbered]{algorithm2e}

\usepackage{picins}
\usepackage{wrapfig}
\usepackage{gastex}
\usepackage{tocloft}
\usepackage[flushmargin]{footmisc} % Fusznoten-Einzug-Patch
\usepackage[tight,nice]{units}

\usepackage{bibunits}
%\usepackage{printlen}

\usepackage{geometry}
\geometry{layoutvoffset=9mm}
\geometry{textwidth=137.07mm}
\geometry{textheight=208.07mm}

\geometry{bindingoffset=16mm}


\usepackage[rightcaption]{sidecap}

% hyperref immer als letztes!
\ifdraft{%
\newcommand{\url}[1]{\texttt{#1}}
}{%
\usepackage[dvips,colorlinks=true,breaklinks=true,hyperindex=true]{hyperref}
\usepackage{breakurl}
}
 
% DEBUGGING STUFF
% ---------------
% Einkommentieren, um nach unreferenzierten
% Labels zu suchen (muss nach hyperref stehen!)
%\usepackage[showrefs,msgs]{refcheck}
%\usepackage{showidx}

% Ligaturen ausschalten
%\DisableLigatures{encoding = *, family = * }

% Silbentrennung ausschalten
%\usepackage[none]{hyphenat}


\graphicspath{{../img/eps/}{../img/ps/}{../img/plt/}{../data/bt/results/img/}{../data/example_net/}}


\parskip1ex
\parindent0em

\newcommand{\ie}{that is\xspace}
\newcommand{\Eg}{For example\xspace}
\newcommand{\eg}{e.\thinspace{}g.\xspace}
\newcommand{\wrt}{with respect to\xspace}
\newcommand{\cf}{cf.\xspace}
\newcommand{\etc}{etc\xspace}

\DeclareMathOperator{\lift}{lift}
\DeclareMathOperator{\conf}{conf}
\DeclareMathOperator{\dom}{dom}
\DeclareMathOperator{\supp}{supp}
\DeclareMathOperator{\relsupp}{rel-supp}
\DeclareMathOperator{\abssupp}{abs-supp}
\DeclareMathOperator{\parents}{parents}
\DeclareMathOperator{\recall}{recall}
\DeclareMathOperator{\spec}{spec}
\DeclareMathOperator{\size}{size}
\DeclareMathOperator{\comp}{comp}
\DeclareMathOperator{\wght}{wght}
\DeclareMathOperator{\avg}{avg}
\DeclareMathOperator{\med}{med}
\DeclareMathOperator{\dev}{dev}
\DeclareMathOperator{\preci}{prec}



\newcommand{\att}[1]{\ensuremath{\mathsf{#1}}\xspace}
\def\bigtimes{\mathop{\raise-.5ex\hbox{\LARGE\rm$\times$}}}

%\newcommand{\llbr}{[\hspace{-0.22em}[}
%\newcommand{\rrbr}{]\hspace{-0.22em}]}

\newcommand{\valof}[1]{\ensuremath\llbracket#1\rrbracket}
%\newcommand{\valof}[1]{\ensuremath\llbr#1\rrbr}
\newcommand{\cov}[1]{\ensuremath\bigl[#1\bigr]}

% Zahlenbereiche
%\newcommand{\Integer}[0]{\mathrm{Z\hspace{-0.4em}Z}}
\newcommand{\Natural}[0]{\mathrm{I\hspace{-0.8mm}N}}
\newcommand{\Real}[0]{\mathrm{I\hspace{-0.8mm}R}}

\newcommand{\1}[0]{\mathbbmss{1}\!}

\newcommand{\argmax}[0]{\operatornamewithlimits{arg\,max}}

\newcommand{\fig}[1]{Figure~\ref{#1}\xspace}
\newcommand{\Fig}[1]{Figure~\ref{#1}\xspace}
\newcommand{\tab}[1]{Table~\ref{#1}\xspace}
\newcommand{\Tab}[1]{Table~\ref{#1}\xspace}
\newcommand{\alg}[1]{Algorithm~\ref{#1}\xspace}
\newcommand{\Alg}[1]{Algorithm~\ref{#1}\xspace}
\newcommand{\sect}[1]{Section~\ref{#1}\xspace}
\newcommand{\Sect}[1]{Section~\ref{#1}\xspace}
\newcommand{\defn}[1]{Definition~\ref{#1}\xspace}
\newcommand{\Defn}[1]{Definition~\ref{#1}\xspace}
\newcommand{\eq}[1]{equation~\ref{#1}\xspace}
\newcommand{\Eq}[1]{Equation~\ref{#1}\xspace}
\newcommand{\Chap}[1]{Chapter~\ref{#1}\xspace}
\newcommand{\chap}[1]{Chapter~\ref{#1}\xspace}
\newcommand{\code}[1]{\texttt{#1}}

\newcommand{\fsd}[2]{\mu_{\Delta_{\mathrm{#1}}}^{\ling{#2}}\xspace}
\newcommand{\fs}[2]{\mu_{#1}^{\ling{#2}}\xspace}
\newcommand{\ling}[1]{\ensuremath{\mathsf{#1}}}

\newcommand{\?}[0]{\texttt{?}}
\newcommand{\fcis}[0]{\mathrm{~is~}}
\newcommand{\fcand}[0]{\mathrm{~and~}}
\newcommand{\fcor}[0]{\mathrm{~or~}}
\newcommand{\fcnot}[0]{\mathrm{not~}}
\newcommand{\notion}[1]{\textsc{#1}}
\newcommand{\etal}{et\thinspace{}al.\xspace}

\newcommand{\ob}[1]{\ovalbox{\sf #1}}

% "\headheight is too small"-Warnung loswerden
\setlength{\headheight}{15pt}

% \tocless verhindert die Aufnahme ins IHV
% (behaelt aber Nummerierung bei)
%-----------------------------------------------------------------------------
\newcommand{\nocontentsline}[3]{}
\newcommand{\tocless}[2]{\bgroup\let\addcontentsline=\nocontentsline#1{#2}\egroup}

%\renewcommand{\emph}[1]{\textbf{#1}}


%----------------------------------------------------------
\newtheorem{theorem}{Theorem}
%\newtheorem{acknowledgement}[theorem]{Acknowledgement}
%\newtheorem{algorithm}[theorem]{Algorithm}
%\newtheorem{axiom}[theorem]{Axiom}
%\newtheorem{case}[theorem]{Case}
%\newtheorem{claim}[theorem]{Claim}
%\newtheorem{conclusion}[theorem]{Conclusion}
%\newtheorem{condition}[theorem]{Condition}
%\newtheorem{conjecture}[theorem]{Conjecture}
%\newtheorem{corollary}[theorem]{Corollary}
%\newtheorem{criterion}[theorem]{Criterion}
\newtheorem{definition}[theorem]{Definition}
%\newtheorem{example}[theorem]{Example}
%\newtheorem{exercise}[theorem]{Exercise}
%\newtheorem{lemma}[theorem]{Lemma}
%\newtheorem{notation}[theorem]{Notation}
%\newtheorem{problem}[theorem]{Problem}
%\newtheorem{proposition}[theorem]{Proposition}
%\newtheorem{remark}[theorem]{Remark}
%\newtheorem{solution}[theorem]{Solution}
%\newtheorem{summary}[theorem]{Summary}
%\newenvironment{proof}[1][Proof]{\textbf{#1.} }{\ \rule{0.5em}{0.5em}}

% Farbdefinitionen
%-----------------------------------------------------------------------------
\definecolor{dark-red}{rgb}{.6, 0, 0}
\definecolor{dark-blue}{rgb}{0, 0, .6}
\definecolor{dark-green}{rgb}{0, .5, 0}
\definecolor{lavender}{rgb}{.902, .902, .98}


% Setup fuer Links
%-----------------------------------------------------------------------------
\ifdraft{}{%
\hypersetup{%
	pdftitle = {Discovery and Visualization of Interesting Patterns},
	pdfauthor = {Matthias Steinbrecher},
	%citecolor={dark-blue},
	%linkcolor={dark-red},
	%urlcolor={dark-green}
	citecolor={black},
	linkcolor={black},
	urlcolor={black}
}
}

% Fusznoten innerhalb von Tabellen sollen angezeigt werden
\makesavenoteenv{tabular}



\newcounter{baustellen}
\setcounter{baustellen}{0}
\newcommand{\baustelle}[1]{%
	\addtocounter{baustellen}{1}%
	\marginpar{%
		\begin{flushleft}
		\includegraphics[width=1.5cm]{baustelle1}\\%
	 		{\color{red}\footnotesize TODO \#\thebaustellen}\linebreak%
			{\color{red}\scriptsize #1}%
	 	\end{flushleft}%
	}%
}
\renewcommand{\baustelle}[1]{}

% Fontanpassung und vertikaler Platz vor den Ueberschriften 
% der Verzeichnisse (damit sie auf gleicher Hoehe wie die 
% Chapter mit groszen Zahlen sind)
%-----------------------------------------------------------------------------
\setlength\cftbeforetoctitleskip{25.5mm}
\setlength\cftbeforeloftitleskip{25.5mm}
\setlength\cftbeforelottitleskip{25.5mm}
\renewcommand{\cfttoctitlefont}{\hfill\huge}
\renewcommand{\cftloftitlefont}{\hfill\huge}
\renewcommand{\cftlottitlefont}{\hfill\huge}


% Sorgt dafuer, dass die mit dem Package footmisc einzugslosen Fusznotennummern
% nicht auszerhalb des linken Randes sind....
%-----------------------------------------------------------------------------
\makeatletter
\newlength{\myFootnoteWidth}
\newlength{\myFootnoteLabel}
\setlength{\myFootnoteLabel}{0.7em}%  <-- can be changed to any valid value
\renewcommand{\@makefntext}[1]{%
  \setlength{\myFootnoteWidth}{\columnwidth}%
  \addtolength{\myFootnoteWidth}{-\myFootnoteLabel}%
  \noindent\makebox[\myFootnoteLabel][r]{\@makefnmark\ }%
  \parbox[t]{\myFootnoteWidth}{#1}%
}
\makeatother


% Definitions from Graphical Models book
\definecolor{LightGray}{rgb}{.8,.8,.8}
\def\support{\mathop{\rm support}\nolimits}
\def\myfrac#1#2{{}^{#1}\kern-.3ex/\kern-.25ex{}_{#2}}
\newsavebox{\tmpboxa}

\def\cbox(#1,#2)#3{\put(#1,#2){% centered text in 10 x 6 box
  \vbox to 6\unitlength{\vss
  \hbox to10\unitlength{\hss#3\hss}\vss}}}

\def\csqr(#1,#2)#3{\put(#1,#2){% centered text in 10 x 10 box
  \vbox to10\unitlength{\vss
  \hbox to10\unitlength{\hss#3\hss}\vss}}}


\clubpenalty = 10000 
\widowpenalty = 10000 
\displaywidowpenalty = 10000


% Silbentrennung
%-----------------------------------------------------------------------------
\hyphenation{Ag-ra-wal}
\hyphenation{Koun-dou-ra-kis}
\hyphenation{Theo-dou-li-dis}
\hyphenation{Goet-hals}
\hyphenation{Asun-cion}
\hyphenation{Bor-gelt}
\hyphenation{Spie-gel-hal-ter}

\hyphenation{data-base}
\hyphenation{bet-ween}
\hyphenation{para-mount}
\hyphenation{an-te-ce-dent}
\hyphenation{an-te-ce-dents}

\hyphenation{Unter-neh-men}
\hyphenation{Da-ten-meng-en}
\hyphenation{re-agie-ren}
\hyphenation{Ana-ly-se-metho-den}
\hyphenation{Aus-wert-ung-en}


\bibpunct[, ]{[}{]}{,}{a}{}{,}
\bibliographystyle{../bib/noon}



% Einheitliche Schreibungen

% data set
% item set
% rule set
% time series
% time frame
% time stamp
% time span

% database

% antecedent
% consequent

% focused
% labeled
% modeled 

% behavior
% color

% (in)dependence
% (in)dependences

% real-world

% gray 

% above-mentioned
% counter-clockwise

% $N$ ... |U|
% $M$ ... set of all measures
% $T$ ... number of time frames
% $R$ ... set of all rules
% $D$ ... database
% $X$ ... antecedent
% $Y$ ... consequent
% $I$ ... item set
% $J$ ... item set
% $\Xi$ . set of all fuzzy concepts
 
