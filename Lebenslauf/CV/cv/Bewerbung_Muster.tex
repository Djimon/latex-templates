\documentclass[a4paper,11pt]{book} %draft%

\usepackage{amsmath,amsfonts,amssymb,bbm}	
\usepackage{graphicx}						% Einbinden von Bildern
\usepackage{phdthesis}						% Styledatei
\usepackage[utf8]{inputenc}					% Umlaute
\usepackage{epstopdf}						% einbinden von .eps Dateien in pdflatex, erfordert ghostscript
\usepackage{psboxit}
\usepackage[ngerman]{babel}					% Formatierungen in deutscher Sprache z.B. Datum

\usepackage{blindtext}						% Lorem Ipsum
\usepackage[colorinlistoftodos]{todonotes} 	%\todo, \missingfigure und \listoftodos (siehe unten für eigene Definitionen)


\usepackage[sort]{natbib}					% Zitationstyle
\usepackage{bibunits}						% Einbinden mehrerer Literaturverzeichnisse


\usepackage{nicefrac}						% schönere Darstellung von Brüchen
\usepackage{subcaption}						% Bildunterschriften mehrerer Subfigures 
\usepackage{multirow}						% Tabellen mit Zeilen und Spaltenverbunden
\usepackage[]{caption}						% Bildunterschriften bei mehreren Teilgrafiken in einer Figure
\usepackage{catoptions}						


% EIGENE (Christoph Dollase)
\usepackage{csquotes}
\usepackage{tcolorbox}
\newtcolorbox
[auto counter,number within=section]{bsp}[2][]{
	colback=black!5!black,colframe=black!40!white,fonttitle=\bfseries,
	title=Bsp.~
	\thetcbcounter
	: #2,#1}
%\newtcolorbox[auto counter,number within=chapter]{bsp}[1][]{
%	fonttitle=\scshape,
%	title={Bsp. \thetcbcounter},
%	#1}

% Todo Notationen
% -----------------------------------------------------------
\newcommand{\newtodo}[1]{\todo[inline, color=yellow!90]{#1}}
\newcommand{\wichtig}[1]{\todo[inline, color=red!65]{#1}}  
\newcommand{\reread}[1]{\todo[color=green!90]{#1}}
\newcommand{\change}[1]{\todo[color=blue!40]{#1}}
\newcommand{\info}[1]{\todo[color=white, bordercolor=black]{#1}}


% ENDE EIGENE

% Schriftart und Textformatierungen
% -----------------------------------------------------------
\usepackage{fourier}        
\DeclareMathAlphabet{\mathcal}{OMS}{cmsy}{m}{n}
\usepackage{scalefnt}
\usepackage[scaled=0.875]{helvet} % ss
\renewcommand{\ttdefault}{lmtt} %tt
\usepackage{microtype}						% Verbesserungen im Texfluss
\setlength{\emergencystretch}{1em}

\DeclareFontFamily{U}{rcjhbltx}{}
\DeclareFontShape{U}{rcjhbltx}{m}{n}{<->rcjhbltx}{}
\DeclareSymbolFont{hebrewletters}{U}{rcjhbltx}{m}{n}
\DeclareMathSymbol{\ayin}{\mathord}{hebrewletters}{96}
\DeclareMathSymbol{\beth}{\mathord}{hebrewletters}{98}\let\bet\beth


% Literaturverzeichnis und Inhaltslisten (Bilder, Tabellen, Algorithmen)
% ----------------------------------------------------------
\usepackage{tocloft}						% Beeinflussen des Literaturverzeichnisses und anderer Inhaltslisten


% Tabellenformatierung
% ----------------------------------------------------------
\usepackage{pgfplotstable}
\usepackage{booktabs}
\usepackage{slashbox}

% EIGENE Tabellen anpassung
\usepackage{tabularx}
\newcolumntype{L}[1]{>{\raggedright\arraybackslash}p{#1}} % linksbündig mit Breitenangabe
\newcolumntype{C}[1]{>{\centering\arraybackslash}p{#1}} % zentriert mit Breitenangabe
\newcolumntype{R}[1]{>{\raggedleft\arraybackslash}p{#1}} % rechtsbündig mit Breitenangabe
% ENDE

% Schusterjungen und Hurenkinder bestrafen
% -----------------------------------------------------------
\clubpenalty = 10000 
\widowpenalty = 10000 
\displaywidowpenalty = 10000


% Darstellungen von Graphen
% -----------------------------------------------------------
\usepackage{tikz}

%\usepackage[pdfborder	={0 0 0}]{hyperref}
\usepackage[hidelinks]{hyperref}


\makeatletter
\pgfdeclarelayer{background}
\pgfdeclarelayer{foreground}
\pgfsetlayers{background,main,foreground}



% Algorithmen und Pseudocode
% -----------------------------------------------------------
\usepackage{algorithmic}
\usepackage{algorithm} 
\renewcommand{\listalgorithmname}{Algorithmenverzeichnis}
\floatname{algorithm}{Algorithmus} 
\renewcommand{\algorithmicrequire}{\textbf{Eingabe:}} 
\renewcommand{\algorithmicensure}{\textbf{Ausgabe:}} 
\renewcommand{\algorithmicreturn}{\textbf{Rückgabe:}} 
%\renewcommand{\algorithmifloatname}{\textbf{Algorithmus}} 
\renewcommand{\algorithmicwhile}{\textbf{So lange}} 
\renewcommand{\algorithmicforall}{\textbf{Für alle}} 
\renewcommand{\algorithmicif}{\textbf{Wenn}} 
\renewcommand{\algorithmicthen}{\textbf{dann}} 
\renewcommand{\algorithmicendif}{\textbf{Wenn Ende}} 
\renewcommand{\algorithmicdo}{\textbf{führe aus}} 
\renewcommand{\algorithmicendfor}{\textbf{Für alle Ende}} 
\renewcommand{\algorithmicendwhile}{\textbf{So lange Ende}} 


\parskip1ex
\parindent0em


% Notationen
% -----------------------------------------------------------
\DeclareMathOperator{\lift}{lift}
\newcommand{\indep}{\rotatebox[origin=c]{90}{$\models$}}


% Zahlenbereiche
% -----------------------------------------------------------
\newcommand{\Integer}[0]{\mathrm{Z\hspace{-0.4em}Z}}
\newcommand{\Natural}[0]{\mathrm{I\hspace{-0.8mm}N}}
\newcommand{\Real}[0]{\mathrm{I\hspace{-0.8mm}R}}


% "\headheight is too small"-Warnung loswerden
\setlength{\headheight}{15pt}


% \tocless verhindert die Aufnahme ins IHV
% (behaelt aber Nummerierung bei)
%-----------------------------------------------------------------------------
\newcommand{\nocontentsline}[3]{}
\newcommand{\tocless}[2]{\bgroup\let\addcontentsline=\nocontentsline#1{#2}\egroup}



% Umgebungsdefinitionen
%----------------------------------------------------------
\newtheorem{theorem}{Theorem}
%\newtheorem{acknowledgement}[theorem]{Acknowledgement}
%\newtheorem{algorithm}[theorem]{Algorithm}
%\newtheorem{axiom}[theorem]{Axiom}
%\newtheorem{case}[theorem]{Case}
%\newtheorem{claim}[theorem]{Claim}
%\newtheorem{conclusion}[theorem]{Conclusion}
%\newtheorem{condition}[theorem]{Condition}
%\newtheorem{conjecture}[theorem]{Conjecture}
%\newtheorem{corollary}[theorem]{Corollary}
%\newtheorem{criterion}[theorem]{Criterion}
\newtheorem{definition}[theorem]{Definition}
%\newtheorem{example}[theorem]{Example}
%\newtheorem{exercise}[theorem]{Exercise}
%\newtheorem{lemma}[theorem]{Lemma}
%\newtheorem{notation}[theorem]{Notation}
%\newtheorem{problem}[theorem]{Problem}
%\newtheorem{proposition}[theorem]{Proposition}
%\newtheorem{remark}[theorem]{Remark}
%\newtheorem{solution}[theorem]{Solution}
%\newtheorem{summary}[theorem]{Summary}
%\newenvironment{proof}[1][Proof]{\textbf{#1.} }{\ \rule{0.5em}{0.5em}}


% Farbdefinitionen
%-----------------------------------------------------------------------------
\definecolor{blue}{RGB}{102,102,255}
\definecolor{green}{RGB}{152,223,138}
\definecolor{violet}{RGB}{148,103,189}
\definecolor{pablue}{RGB}{0,77,128}
\definecolor{palightlue}{RGB}{0,143,213}
\definecolor{pagreen}{RGB}{153,204,51}
\definecolor{paorange}{RGB}{255,153,51}



% Sorgt dafuer, dass die mit dem Package footmisc einzugslosen Fusznotennummern
% nicht auszerhalb des linken Randes sind....
% -----------------------------------------------------------
\makeatletter
\newlength{\myFootnoteWidth}
\newlength{\myFootnoteLabel}
\setlength{\myFootnoteLabel}{0.7em}%  <-- can be changed to any valid value
\renewcommand{\@makefntext}[1]{%
  \setlength{\myFootnoteWidth}{\columnwidth}%
  \addtolength{\myFootnoteWidth}{-\myFootnoteLabel}%
  \noindent\makebox[\myFootnoteLabel][r]{\@makefnmark\ }%
  \parbox[t]{\myFootnoteWidth}{#1}%
}

% capitalized reference
\def\Autoref#1{%
	\begingroup
	\edef\reserved@a{\cpttrimspaces{#1}}%
	\ifcsndefTF{r@#1}{%
		\xaftercsname{\expandafter\testreftype\@fourthoffive}
		{r@\reserved@a}.\\{#1}%
	}{%
	\ref{#1}%
}%
\endgroup
}
\def\testreftype#1.#2\\#3{%
	\ifcsndefTF{#1autorefname}{%
		\def\reserved@a##1##2\@nil{%
			\uppercase{\def\ref@name{##1}}%
			\csn@edef{#1autorefname}{\ref@name##2}%
			\autoref{#3}%
		}%
		\reserved@a#1\@nil
	}{%
	\autoref{#3}%
}%
}

% Definition von Mehrfachreferenzen
\newcommand\multref[1]{\@first@ref#1,@}
\def\@throw@dot#1.#2@{#1}% discard everything after the dot
\def\@set@refname#1{%    % set \@refname to autoefname+s using \getrefbykeydefault
	\edef\@tmp{\getrefbykeydefault{#1}{anchor}{}}%
	\def\@refname{\@nameuse{\expandafter\@throw@dot\@tmp.@autorefname}s}%
}
\def\@first@ref#1,#2{%
	\ifx#2@\autoref{#1}\let\@nextref\@gobble% only one ref, revert to normal \autoref
	\else%
	\@set@refname{#1}%  set \@refname to autoref name
	\@refname~\ref{#1}% add autoefname and first reference
	\let\@nextref\@next@ref% push processing to \@next@ref
	\fi%
	\@nextref#2%
}
\def\@next@ref#1,#2{%
	\ifx#2@ and~\ref{#1}\let\@nextref\@gobble% at end: print and+\ref and stop
	\else, \ref{#1}% print  ,+\ref and continue
	\fi%
	\@nextref#2%
}

% roman numbers
\newcommand*{\rom}[1]{\expandafter\@slowromancap\romannumeral #1@}
\makeatother


% Kürzel für Fließtexte
% -----------------------------------------------------------
\newcommand{\etc}{etc.\xspace}

% Befehle für Matheumgebungt
% -----------------------------------------------------------
\DeclareMathOperator*{\argmin}{arg\,min}

% Befehle für Matheumgebung und Fließtext
% -----------------------------------------------------------
\newcommand{\epsneighborhood}{\ensuremath{\epsilon}-neigh\-bor\-hood\xspace}

% Silbentrennung (speziell für Namen relevant)
% -----------------------------------------------------------
\hyphenation{Ag-ra-wal}
 



%\begin{document}
\begin{document}
%----------------------------------------------------------------------------------------
%	COVER LETTER
%----------------------------------------------------------------------------------------

% To remove the cover letter, comment out this entire block

\clearpage
Christoph Dollase \\
Ottenbergstraße 42,39106 Magdeburg\\
bewerbung@dollase.de \\
+49 174 1913083

\vspace{0,5cm}
% TODO: ADDRESSE DES EMPFÄNGERS 
\recipient{Personalabteilung}{Strehlow GmbH,\\ Havelstra\ss e 23\\ 39126 Magdeburg} % Letter recipient
\date{\today} % Letter date
\opening{} % Opening greeting
\closing{Beste Gr\"u\ss e,} % TODO: Abschluss Grußformel
\enclosure[im Anhang]{Lebenslauf,Zeugnisse} % List of enclosed documents

\makelettertitle % Print letter title

% BEZUG
\textbf{Bezug: Stellenangebot Business Intelligence Professional m/w/d}\\
\vspace{0,5cm}\\
Sehr geehrte Damen und Herren, \\ \vspace{0.5cm}

als Business Intelligence Consultant berate ich mittelst\"andische Unternehmen in Deutschland, Polen, Groß-Britannien und den USA in unserer hauseigenen Business Intelligence Lösung. Dabei begleite ich meine Kunden von der Konzeption über die Einführung und Schulung bis hin zur Feinoptimierung der Analysen. Ich wurde glücklicherweise in der kompletten Klaviatur des BI-Business (Architektur, Installation, ETL, Konzeption, Entwicklung, Design, Schulung) ausgebildet. Mein aktueller Arbeitgeber - proALPHA GmbH - ist ein ERP-Dienstleister mit einem vollumfänglichen ERP-System, das unsere Hauptdatenquelle im BI-Bereich darstellt. Durch die heterogene Kundenlandschaft (Großhandel, Kunststoff, Maschinenbau, Alu) habe ich Erfahrungen in Logistik, Materialfluss, Produktion und vor allem in meiner Spezialisierung Finanzen und Controlling sammeln können. Die Kombination aus ERP-Wissen und BI-Expertise stattet mich ideal aus, um Ihren Bedürfnissen zu entsprechen.\\
In unserem Team genie\ss en wir die Freiheit unsere Aufgabenpakete und Projekte komplett eigenst\"andig zu organisieren und zu planen. Durch eine dynamische Zeitgestaltung haben wir eine sehr harmonische Arbeitsumgebung geschaffen, in der wir uns gegenseitig tagt\"aglich aushelfen und unser Wissen breitflächig im Team aufbauen können. Dabei habe ich die Erfahrung gemacht, dass sich eine ausgedehnte Strukturierungs- und Planungsphase hinten heraus auszahlt und zus\"atzliche Schleifen vermeidet.\\
Womit ich bei meinen Kunden sehr gut ankomme, ist die Abstraktion verschieden komplexer Gesch\"aftsprozesse auf einfache Modelle. Mein persönliches Steckenpferd ist vor allem "Design \& Usability" (anwenderfreundliche L\"osungen). Anstatt stur gegebene Anforderungen umzusetzen, animiere ich die KEY-User dazu, sich modernen Visualisierungen zu öffnen und sich von riesigen Excel-Ausdrucken zu trennen. 
Als Mitglied unserer internen Produktentwicklung sind mir verschiedene Ansätze zu Data-Warehouse-Architektur und automatisierten ETL-Prozessen mit QlikView und QlikSense vertraut.
Da ich erst 2 Jahre im aktuellen Team bin, kann ich ein Arbeitszeugnis erst nach Beendigung des aktuellen Verhältnisses nachreichen.

In meiner Position als studentischer \"Ubungsleiter für Intelligente Systeme (Einführung in KI) konnte ich die verschiedenen Konzepte des Machine Learning und allgemeine KI-Ansätze vertiefen. Daher habe ich auch eine hohe intrinsische Motivation verfügbare Massendaten mit entsprechenden KI-Verfahren anzureichern.

Ich freue mich auf eine baldige Antwort von Ihnen, um m\"oglichst zeitnah die Stelle antreten zu k\"onnen.
 
%\vspace{1cm}
\makeletterclosing % Print letter signature

\newpage



%----------------------------------------------------------------------------------------
%	CURRICULUM VITAE
%----------------------------------------------------------------------------------------
%\documentclass[a4paper,11pt]{book} %draft%

\usepackage{amsmath,amsfonts,amssymb,bbm}	
\usepackage{graphicx}						% Einbinden von Bildern
\usepackage{phdthesis}						% Styledatei
\usepackage[utf8]{inputenc}					% Umlaute
\usepackage{epstopdf}						% einbinden von .eps Dateien in pdflatex, erfordert ghostscript
\usepackage{psboxit}
\usepackage[ngerman]{babel}					% Formatierungen in deutscher Sprache z.B. Datum

\usepackage{blindtext}						% Lorem Ipsum
\usepackage[colorinlistoftodos]{todonotes} 	%\todo, \missingfigure und \listoftodos (siehe unten für eigene Definitionen)


\usepackage[sort]{natbib}					% Zitationstyle
\usepackage{bibunits}						% Einbinden mehrerer Literaturverzeichnisse


\usepackage{nicefrac}						% schönere Darstellung von Brüchen
\usepackage{subcaption}						% Bildunterschriften mehrerer Subfigures 
\usepackage{multirow}						% Tabellen mit Zeilen und Spaltenverbunden
\usepackage[]{caption}						% Bildunterschriften bei mehreren Teilgrafiken in einer Figure
\usepackage{catoptions}						


% EIGENE (Christoph Dollase)
\usepackage{csquotes}
\usepackage{tcolorbox}
\newtcolorbox
[auto counter,number within=section]{bsp}[2][]{
	colback=black!5!black,colframe=black!40!white,fonttitle=\bfseries,
	title=Bsp.~
	\thetcbcounter
	: #2,#1}
%\newtcolorbox[auto counter,number within=chapter]{bsp}[1][]{
%	fonttitle=\scshape,
%	title={Bsp. \thetcbcounter},
%	#1}

% Todo Notationen
% -----------------------------------------------------------
\newcommand{\newtodo}[1]{\todo[inline, color=yellow!90]{#1}}
\newcommand{\wichtig}[1]{\todo[inline, color=red!65]{#1}}  
\newcommand{\reread}[1]{\todo[color=green!90]{#1}}
\newcommand{\change}[1]{\todo[color=blue!40]{#1}}
\newcommand{\info}[1]{\todo[color=white, bordercolor=black]{#1}}


% ENDE EIGENE

% Schriftart und Textformatierungen
% -----------------------------------------------------------
\usepackage{fourier}        
\DeclareMathAlphabet{\mathcal}{OMS}{cmsy}{m}{n}
\usepackage{scalefnt}
\usepackage[scaled=0.875]{helvet} % ss
\renewcommand{\ttdefault}{lmtt} %tt
\usepackage{microtype}						% Verbesserungen im Texfluss
\setlength{\emergencystretch}{1em}

\DeclareFontFamily{U}{rcjhbltx}{}
\DeclareFontShape{U}{rcjhbltx}{m}{n}{<->rcjhbltx}{}
\DeclareSymbolFont{hebrewletters}{U}{rcjhbltx}{m}{n}
\DeclareMathSymbol{\ayin}{\mathord}{hebrewletters}{96}
\DeclareMathSymbol{\beth}{\mathord}{hebrewletters}{98}\let\bet\beth


% Literaturverzeichnis und Inhaltslisten (Bilder, Tabellen, Algorithmen)
% ----------------------------------------------------------
\usepackage{tocloft}						% Beeinflussen des Literaturverzeichnisses und anderer Inhaltslisten


% Tabellenformatierung
% ----------------------------------------------------------
\usepackage{pgfplotstable}
\usepackage{booktabs}
\usepackage{slashbox}

% EIGENE Tabellen anpassung
\usepackage{tabularx}
\newcolumntype{L}[1]{>{\raggedright\arraybackslash}p{#1}} % linksbündig mit Breitenangabe
\newcolumntype{C}[1]{>{\centering\arraybackslash}p{#1}} % zentriert mit Breitenangabe
\newcolumntype{R}[1]{>{\raggedleft\arraybackslash}p{#1}} % rechtsbündig mit Breitenangabe
% ENDE

% Schusterjungen und Hurenkinder bestrafen
% -----------------------------------------------------------
\clubpenalty = 10000 
\widowpenalty = 10000 
\displaywidowpenalty = 10000


% Darstellungen von Graphen
% -----------------------------------------------------------
\usepackage{tikz}

%\usepackage[pdfborder	={0 0 0}]{hyperref}
\usepackage[hidelinks]{hyperref}


\makeatletter
\pgfdeclarelayer{background}
\pgfdeclarelayer{foreground}
\pgfsetlayers{background,main,foreground}



% Algorithmen und Pseudocode
% -----------------------------------------------------------
\usepackage{algorithmic}
\usepackage{algorithm} 
\renewcommand{\listalgorithmname}{Algorithmenverzeichnis}
\floatname{algorithm}{Algorithmus} 
\renewcommand{\algorithmicrequire}{\textbf{Eingabe:}} 
\renewcommand{\algorithmicensure}{\textbf{Ausgabe:}} 
\renewcommand{\algorithmicreturn}{\textbf{Rückgabe:}} 
%\renewcommand{\algorithmifloatname}{\textbf{Algorithmus}} 
\renewcommand{\algorithmicwhile}{\textbf{So lange}} 
\renewcommand{\algorithmicforall}{\textbf{Für alle}} 
\renewcommand{\algorithmicif}{\textbf{Wenn}} 
\renewcommand{\algorithmicthen}{\textbf{dann}} 
\renewcommand{\algorithmicendif}{\textbf{Wenn Ende}} 
\renewcommand{\algorithmicdo}{\textbf{führe aus}} 
\renewcommand{\algorithmicendfor}{\textbf{Für alle Ende}} 
\renewcommand{\algorithmicendwhile}{\textbf{So lange Ende}} 


\parskip1ex
\parindent0em


% Notationen
% -----------------------------------------------------------
\DeclareMathOperator{\lift}{lift}
\newcommand{\indep}{\rotatebox[origin=c]{90}{$\models$}}


% Zahlenbereiche
% -----------------------------------------------------------
\newcommand{\Integer}[0]{\mathrm{Z\hspace{-0.4em}Z}}
\newcommand{\Natural}[0]{\mathrm{I\hspace{-0.8mm}N}}
\newcommand{\Real}[0]{\mathrm{I\hspace{-0.8mm}R}}


% "\headheight is too small"-Warnung loswerden
\setlength{\headheight}{15pt}


% \tocless verhindert die Aufnahme ins IHV
% (behaelt aber Nummerierung bei)
%-----------------------------------------------------------------------------
\newcommand{\nocontentsline}[3]{}
\newcommand{\tocless}[2]{\bgroup\let\addcontentsline=\nocontentsline#1{#2}\egroup}



% Umgebungsdefinitionen
%----------------------------------------------------------
\newtheorem{theorem}{Theorem}
%\newtheorem{acknowledgement}[theorem]{Acknowledgement}
%\newtheorem{algorithm}[theorem]{Algorithm}
%\newtheorem{axiom}[theorem]{Axiom}
%\newtheorem{case}[theorem]{Case}
%\newtheorem{claim}[theorem]{Claim}
%\newtheorem{conclusion}[theorem]{Conclusion}
%\newtheorem{condition}[theorem]{Condition}
%\newtheorem{conjecture}[theorem]{Conjecture}
%\newtheorem{corollary}[theorem]{Corollary}
%\newtheorem{criterion}[theorem]{Criterion}
\newtheorem{definition}[theorem]{Definition}
%\newtheorem{example}[theorem]{Example}
%\newtheorem{exercise}[theorem]{Exercise}
%\newtheorem{lemma}[theorem]{Lemma}
%\newtheorem{notation}[theorem]{Notation}
%\newtheorem{problem}[theorem]{Problem}
%\newtheorem{proposition}[theorem]{Proposition}
%\newtheorem{remark}[theorem]{Remark}
%\newtheorem{solution}[theorem]{Solution}
%\newtheorem{summary}[theorem]{Summary}
%\newenvironment{proof}[1][Proof]{\textbf{#1.} }{\ \rule{0.5em}{0.5em}}


% Farbdefinitionen
%-----------------------------------------------------------------------------
\definecolor{blue}{RGB}{102,102,255}
\definecolor{green}{RGB}{152,223,138}
\definecolor{violet}{RGB}{148,103,189}
\definecolor{pablue}{RGB}{0,77,128}
\definecolor{palightlue}{RGB}{0,143,213}
\definecolor{pagreen}{RGB}{153,204,51}
\definecolor{paorange}{RGB}{255,153,51}



% Sorgt dafuer, dass die mit dem Package footmisc einzugslosen Fusznotennummern
% nicht auszerhalb des linken Randes sind....
% -----------------------------------------------------------
\makeatletter
\newlength{\myFootnoteWidth}
\newlength{\myFootnoteLabel}
\setlength{\myFootnoteLabel}{0.7em}%  <-- can be changed to any valid value
\renewcommand{\@makefntext}[1]{%
  \setlength{\myFootnoteWidth}{\columnwidth}%
  \addtolength{\myFootnoteWidth}{-\myFootnoteLabel}%
  \noindent\makebox[\myFootnoteLabel][r]{\@makefnmark\ }%
  \parbox[t]{\myFootnoteWidth}{#1}%
}

% capitalized reference
\def\Autoref#1{%
	\begingroup
	\edef\reserved@a{\cpttrimspaces{#1}}%
	\ifcsndefTF{r@#1}{%
		\xaftercsname{\expandafter\testreftype\@fourthoffive}
		{r@\reserved@a}.\\{#1}%
	}{%
	\ref{#1}%
}%
\endgroup
}
\def\testreftype#1.#2\\#3{%
	\ifcsndefTF{#1autorefname}{%
		\def\reserved@a##1##2\@nil{%
			\uppercase{\def\ref@name{##1}}%
			\csn@edef{#1autorefname}{\ref@name##2}%
			\autoref{#3}%
		}%
		\reserved@a#1\@nil
	}{%
	\autoref{#3}%
}%
}

% Definition von Mehrfachreferenzen
\newcommand\multref[1]{\@first@ref#1,@}
\def\@throw@dot#1.#2@{#1}% discard everything after the dot
\def\@set@refname#1{%    % set \@refname to autoefname+s using \getrefbykeydefault
	\edef\@tmp{\getrefbykeydefault{#1}{anchor}{}}%
	\def\@refname{\@nameuse{\expandafter\@throw@dot\@tmp.@autorefname}s}%
}
\def\@first@ref#1,#2{%
	\ifx#2@\autoref{#1}\let\@nextref\@gobble% only one ref, revert to normal \autoref
	\else%
	\@set@refname{#1}%  set \@refname to autoref name
	\@refname~\ref{#1}% add autoefname and first reference
	\let\@nextref\@next@ref% push processing to \@next@ref
	\fi%
	\@nextref#2%
}
\def\@next@ref#1,#2{%
	\ifx#2@ and~\ref{#1}\let\@nextref\@gobble% at end: print and+\ref and stop
	\else, \ref{#1}% print  ,+\ref and continue
	\fi%
	\@nextref#2%
}

% roman numbers
\newcommand*{\rom}[1]{\expandafter\@slowromancap\romannumeral #1@}
\makeatother


% Kürzel für Fließtexte
% -----------------------------------------------------------
\newcommand{\etc}{etc.\xspace}

% Befehle für Matheumgebungt
% -----------------------------------------------------------
\DeclareMathOperator*{\argmin}{arg\,min}

% Befehle für Matheumgebung und Fließtext
% -----------------------------------------------------------
\newcommand{\epsneighborhood}{\ensuremath{\epsilon}-neigh\-bor\-hood\xspace}

% Silbentrennung (speziell für Namen relevant)
% -----------------------------------------------------------
\hyphenation{Ag-ra-wal}
 



%\begin{document}

%----------------------------------------------------------------------------------------
%	COVER LETTER
%----------------------------------------------------------------------------------------

% To remove the cover letter, comment out this entire block




%----------------------------------------------------------------------------------------
%	CURRICULUM VITAE
%----------------------------------------------------------------------------------------

\makecvtitle % Print the CV title

%----------------------------------------------------------------------------------------
%	EDUCATION SECTION
%----------------------------------------------------------------------------------------
%\vspace{-1cm}
\section{Berufserfahrung}

\subsection{Hauptt\"atigkeiten}

\cventry{Okt 2018 -- jetzt}{Business Intelligence Consultant}{\textsc{proALPHA Consulting GmbH}}{Barleben}{Business Intelligence Competence Center}
{
	\begin{itemize}
		\item Management-Beratung und Schulung der proALPHA BI-L\"osungen. 
		\item Konzeptionierung und Umsetzung komplexer Analysen und Auswertungen mit QlikView und QlikSense; Datenmodellierung; Datawarehousing 
		\item Enger Kundenkontakt und Begleitung der Key-User von der Einf\"uhrung bis zur eigenst\"andigen Entwicklung von Analysen im Self-Service-BI. 
		\item Teilprojektleiter in internationalen Projekten mit Kunden aus Deutschland, Polen, Gro\ss britannien und den USA. 
	\end{itemize}}

%------------------------------------------------
\cventry{Apr 2016 -- Okt 2018}{Softwareentwickler (Dualstudent)}{\textsc{proALPHA Consulting GmbH}}{Barleben}{Competence Center Technology}
{ 
	\begin{itemize} 
		\item Wartung und Erweiterung Business Intelligence Backend
		\item Bachelorarbeit im Unternehmen
		\item ERP Lagerintegration auf dem Smartphone\\ \textsc{(Sieger des firmeninternen Innovations-Wettbewerbs)}.
		\item Wartung und Erweiterung Datenmigrations-Tool. 
	\end{itemize}}

\cventry{Okt 2014 -- Apr 2016}{Softwareentwickler (Dualstudent)}{\textsc{isM Systemtechnk GmbH}}{Barleben}{}{
	\begin{itemize} 
		\item Datenbankoptimierung, Triggerprogrammierung.
		\item \"Ubernahme durch proALPHA Consulting GmbH (Apr 2016)
	\end{itemize}}
%------------------------------------------------

\subsection{Nebent\"atigkeiten}

\cventry{Okt 2017 -- Feb 2019}{\"Ubungsleiter}{\textsc{Otto-von-Guericke-Universit\"at, Magdeburg}}{IKS}{Arbeitsgruppe Computational Intelligence}
{
	\begin{itemize} 
		\item Lehrt\"atigkeit als studentischer Tutor f\"ur die Pflicht\"ubungen im Modul "Einf\"uhrung in Intelligente Systeme''
		\item Evaluationsergebis durch Studierende: 1,1 
	\end{itemize}}

\cventry{Jan 2011 -- Mai 2014}{Darsteller, Musiker und Tänzer}{\textsc{diverse Projekte}}{deutschlandweit}{}
{
	\begin{itemize} 
		\item Talas Hanse GmbH "Über sieben Brücken" - Musical 
		\item Mehrere Komparsenrollen für Serien und Fernsehfilme (z.B. "Wir waren Könige")
		\item Saxofonist in verschiedenen Ensembels (u.a. Jazz-Band und Musical-Orchester)
\end{itemize}}

%\cventry{Okt 2013 -- Jul 2014}{Freier Mitarbeiter}{\textsc{ZGS Bildungs-GmbH}}{Halle (Saale)}{Sch\"ulerhilfe}
%{
%	\begin{itemize} 
%		\item Nachhilfe SEK I + II (Gymnasium)
%		\item F\"acher: Mathe, Physik, Biologie, Chemie 
%	\end{itemize}}

%\cventry{Oktr 2011 -- Mai 2014}{Darsteller und T\"anzer}{\textsc{TALAS GmbH}}{Stendal}{Musical - "\"Uber sieben Br\"ucken"}
%{
%	\begin{itemize} 
%		\item Auftritte fast jedes Wochenende in verschiedenen St\"adten
%	\end{itemize}}



\section{Bachelor Thesis}

\cvitem{Title}{\emph{Maschinelle Klassifizierung von Pflichtenheften - Eine  Evaluation verschiedener Classifier}}
\cvitem{Betreuer}{Prof. Andreas N\"urnberger \& M.Sc. Thomas Low \& M.Sc. Johannes Schwerdt}
\cvitem{Beschreibung}{In der Arbeit wird untersucht, wie hoch die Erfolgsraten verschiedener Classifier beim maschinellen Lernen von Kategorien in einer realen Umgebung (keine Demodaten) sind.}
\cvitem{Note}{1,3}



\section{Ausbildung}

\cventry{ab 2020}{Master of Science (Wirtschafts-)Informatik}{Otto-von-Guericke-Universit\"at}{Magdeburg}{nebenberuflich in Teilzeit, Start zum WiSe 20/21}{}
\cventry{2014-2018\footnotemark}{Bachelor of Science Informatik}{Otto-von-Guericke-Universit\"at}{Magdeburg}{}{Spezialisierung: k\"unstliche Intelligenz, Nebenfach: Management, \hfill \textit{Abschussnote: 2,4}}  % Arguments not required can be left empty

\cventry{2009-2014}{Bachelor of Science* Biochemie}{Martin-Luther-Universit\"at Halle-Wittenberg}{Halle (Saale)}{Spezialisierung: Genetik,\textit{{*}ohne Abschluss}}{}  % Arguments not required can be left empty

\cventry{2008}{Deutsches Abitur}{Rudolf-Hildebrand-Gymnasium}{Stendal}{}{Leistungskurse: Mathematik, Biologie \textit{\hfill Abschlussnote: 2,2}}

\footnotetext{$^1$um 1 Semester verkürzt, lt.Regelstudienzeit (Informatik Dual: 9 Semester) ursprünglich geplantes Ende war März 2019}

\section{Ehrenamt}

\cvitem{seit M\"ar 2019}{Magdeburgs Studierende e.V. (Gr\"undungsmitglied)}
\cvitem{seit Okt 2014}{Acagamics e.V. (Verein f\"ur Spieleentwicklung, 2 Jahre im Vorstand)}
\cvitem{Jul 2016 -- Jun 2018}{Studierendenrat der Otto-von-Guericke-Universit\"at Magdeburg}
\cvitem{Okt 2014 -- Jun 2018}{Fachschaftsrat der Fakult\"at f\"ur Informatik an der Otto-von-Guericke-Universit\"at Magdeburg (1 Jahr im Vorstand)}


\section{Allgemeine Software und IT-Kenntnisse}


\cvitem{Professionell}{QlikView, QlikSense, BI-Architektur, Informationsvisualisierung}
\cvitem{Sehr Gut}{KI Modellierung; Confluence, Jira und Co.; MS Office; Bild- und Videobearbeitung; Social Media Strategie und Marketing, \LaTeX}
\cvitem{Gut}{Django, Wordpress, SQL}

\subsection{Programmiersprachen}

\cvitem{Sehr Gut}{\textsc{C\#; Python}}
\cvitem{Gut}{\textsc{Java; C++,}}


\section{"Soft-Skills''}

\cvitem{IWP Zertifikat}{Erfolgsorientierte Kommunikations- und Gespr\"achstechniken}
\cvitem{IWP Zertifikat}{Professionelle Gestaltung von Konfliktgespr\"achen}
\cvitem{Bachelor Modul}{"Startup Engineering" - Design-Thinking, Produktinnovation und Business Model Evolution}


\section{Sprachen}

\cvitemwithcomment{Deutsch}{Muttersprache}{}
\cvitemwithcomment{Englisch}{Verhandlungssicher}{}


\section{Interessen}

\renewcommand{\listitemsymbol}{-~} % Changes the symbol used for lists

\cvlistdoubleitem{Saxofon}{Musicals}
\cvlistdoubleitem{Tanzen}{Fotografie und Bildbearbeitung}
\cvlistdoubleitem{Meditation}{digitale Spieleentwicklung}
\cvlistdoubleitem{Aktienhandel}{internationale Bildungssysteme}
\cvlistdoubleitem{Kochen}{Coaching (Speaker)}

\pagebreak

% TODO: Falls FAQ erwünscht, \end{document} auskommentieren







% TODO: Falls FAQ erwünscht, \end{document} auskommentieren
\end{document}

%----------------------------------------------------------------------------------------
% FAQ

 
\huge{Antworten zu den h\"aufigsten Fragen}
\normalsize
\vspace{1cm}

\section{Wieso m\"ochte ich ausgerechnet zu Ihnen?}

\cvitem{}{...}

\section{Wieso bin ich auf der Suche nach einer neuen Herausforderung.}

\cvitem{}{Ich bin mit meiner aktuellen Arbeit sehr zufrieden und helfe viele meiner Kunden dabei ihr BI-Portfolio aufzubauen und ihre Unternehmensentschiedungen zu optimieren. Auch die Arbeit in eiem sehr jungen Team gefällt mir sehr. Es gibt am Status Quo nichts zu klagen. Dennoch hat ein sehr junges Team ein Nachteil. Mir wurde signalisiert, dass die Chance in den nächsten 5 Jahren in eine Position mit Führungsverantwortung zu gelangen, durch die aktuellen Strukturen eher unwahrscheinlich ist.}

\section{Wieso habe ich Biochemie nicht abgeschlossen?}

\cvitem{}{Ich habe nach der H\"alfte des Studiums gemerkt, dass die Arbeit in der Forschung viel stringenter und unfreier ist, als es einem vor dem Studium suggeriert wird. Ich habe dann einige Semester mit Musik (Saxophon), Schauspiel (Komparsenrollen) und als Teil eines Musical-Ensembles meinen Rücklagen aufgebaut. Bis ich dann mutig genug war, das Studium offiziell zu beenden.}

\section{Warum habe ich mich für ein Masterstudium neben dem Beruf entschieden?}

\cvitem{}{Das Aufbaustudium eröffnet zum Einen die Möglichkeit in höhere Managementebenen aufzusteigen, was auf mittlere bis lange Sicht eines meiner Karriereziele ist. Zum Anderen reizt es mich die eine oder andere Veröffentlichung zu verfassen, um am technologischen Fortschritt mitzuwirken.}



\end{document}


